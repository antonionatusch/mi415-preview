%%%%%%%%%%%%%%%%%%%%%%%%%%%%% Define Article %%%%%%%%%%%%%%%%%%%%%%%%%%%%%%%%%%
\documentclass{article}
%%%%%%%%%%%%%%%%%%%%%%%%%%%%%%%%%%%%%%%%%%%%%%%%%%%%%%%%%%%%%%%%%%%%%%%%%%%%%%%

%%%%%%%%%%%%%%%%%%%%%%%%%%%%% Using Packages %%%%%%%%%%%%%%%%%%%%%%%%%%%%%%%%%%
\usepackage{geometry}
\usepackage{float}
\usepackage{biblatex}
\usepackage{csquotes}
\addbibresource{./references/references.bib}
\usepackage[hidelinks]{hyperref}
\usepackage[spanish]{babel}
\usepackage{graphicx}
\usepackage{amssymb}
\usepackage{amsmath}
\usepackage{amsthm}
\usepackage{empheq}
\usepackage{mdframed}
\usepackage{booktabs}
\usepackage{lipsum}
\usepackage{graphicx}
\usepackage{color}
\usepackage{psfrag}
\usepackage{pgfplots}
\usepackage{bm}
\usepackage{enumitem}
%%%%%%%%%%%%%%%%%%%%%%%%%%%%%%%%%%%%%%%%%%%%%%%%%%%%%%%%%%%%%%%%%%%%%%%%%%%%%%%

% Other Settings

%%%%%%%%%%%%%%%%%%%%%%%%%% Page Setting %%%%%%%%%%%%%%%%%%%%%%%%%%%%%%%%%%%%%%%
\geometry{a4paper}

%%%%%%%%%%%%%%%%%%%%%%%%%% Define some useful colors %%%%%%%%%%%%%%%%%%%%%%%%%%
\definecolor{ocre}{RGB}{243,102,25}
\definecolor{mygray}{RGB}{243,243,244}
\definecolor{deepGreen}{RGB}{26,111,0}
\definecolor{shallowGreen}{RGB}{235,255,255}
\definecolor{deepBlue}{RGB}{61,124,222}
\definecolor{shallowBlue}{RGB}{235,249,255}
%%%%%%%%%%%%%%%%%%%%%%%%%%%%%%%%%%%%%%%%%%%%%%%%%%%%%%%%%%%%%%%%%%%%%%%%%%%%%%%

%%%%%%%%%%%%%%%%%%%%%%%%%% Define an orangebox command %%%%%%%%%%%%%%%%%%%%%%%%
\newcommand\orangebox[1]{\fcolorbox{ocre}{mygray}{\hspace{1em}#1\hspace{1em}}}
%%%%%%%%%%%%%%%%%%%%%%%%%%%%%%%%%%%%%%%%%%%%%%%%%%%%%%%%%%%%%%%%%%%%%%%%%%%%%%%

%%%%%%%%%%%%%%%%%%%%%%%%%%%% English Environments %%%%%%%%%%%%%%%%%%%%%%%%%%%%%
\newtheoremstyle{mytheoremstyle}{3pt}{3pt}{\normalfont}{0cm}{\rmfamily\bfseries}{}{1em}{{\color{black}\thmname{#1}~\thmnumber{#2}}\thmnote{\,--\,#3}}
\newtheoremstyle{myproblemstyle}{3pt}{3pt}{\normalfont}{0cm}{\rmfamily\bfseries}{}{1em}{{\color{black}\thmname{#1}~\thmnumber{#2}}\thmnote{\,--\,#3}}
\theoremstyle{mytheoremstyle}
\newmdtheoremenv[linewidth=1pt,backgroundcolor=shallowGreen,linecolor=deepGreen,leftmargin=0pt,innerleftmargin=20pt,innerrightmargin=20pt,]{theorem}{Theorem}[section]
\theoremstyle{mytheoremstyle}
\newmdtheoremenv[linewidth=1pt,backgroundcolor=shallowBlue,linecolor=deepBlue,leftmargin=0pt,innerleftmargin=20pt,innerrightmargin=20pt,]{definition}{Definition}[section]
\theoremstyle{myproblemstyle}
\newmdtheoremenv[linecolor=black,leftmargin=0pt,innerleftmargin=10pt,innerrightmargin=10pt,]{problem}{Problem}[section]
%%%%%%%%%%%%%%%%%%%%%%%%%%%%%%%%%%%%%%%%%%%%%%%%%%%%%%%%%%%%%%%%%%%%%%%%%%%%%%%

%%%%%%%%%%%%%%%%%%%%%%%%%%%%%%% Plotting Settings %%%%%%%%%%%%%%%%%%%%%%%%%%%%%
\usepgfplotslibrary{colorbrewer}
\pgfplotsset{width=8cm,compat=1.9}
%%%%%%%%%%%%%%%%%%%%%%%%%%%%%%%%%%%%%%%%%%%%%%%%%%%%%%%%%%%%%%%%%%%%%%%%%%%%%%%

%%%%%%%%%%%%%%%%%%%%%%%%%%%%%%% Title & Author %%%%%%%%%%%%%%%%%%%%%%%%%%%%%%%%
\title{\textbf{Preguntas Tema 1}}
\author{Antonio Natusch}
%%%%%%%%%%%%%%%%%%%%%%%%%%%%%%%%%%%%%%%%%%%%%%%%%%%%%%%%%%%%%%%%%%%%%%%%%%%%%%%

\begin{document}
\maketitle
\section*{Sobre este documento}
Este documento contiene respuestas a una serie de preguntas vistas en la diapositiva introductoria provista en la materia de Modelación y Simulación.
La misma presentación se encuentra en \href[page=1]{../references/intro.pdf}{el archivo intro.pdf.} % tengo que usar ../ porque el viewer se abre en el directorio build/


\section*{Preguntas del PDF}
\begin{figure}[H]
	\centering
	\includegraphics[page=3,scale=0.9]{./references/intro.pdf}
	\caption{Página 3 de la presentación titulada <<Introducción>>, Modelación y Simulación, Semestre 2/2025.}
\end{figure}

\section*{Respuestas}
\subsection*{Introducción}
\begin{enumerate}
	\item La simulación se refiere a un gran conjunto de métodos y aplicaciones
	      que buscan imitar el comportamiento de sistemas reales, generalmente por medio de
	      una computadora con un software apropiado.
	      Es una herramienta útil en diversas disciplinas ya que, al ser independiente del dominio del problema, es posible
	      recrear situaciones o resultados tantos físicos como teóricos, teniendo su aplicación en sectores como ser la aeronáutica, procesos industriales, etc.
	      \citetitle[p.~2]{garcia2013simpromodel}
	\item Su propósito es crear una representación simplificada y abstracta de un problema real.
	\item En un modelo determinista, las relaciones entre los cambios de las variables del modelo son constantes. En cambio, en un modelo estocástico, los cambios en las variables suelen seguir algún modelo probabilístico donde el resultado después de un evento no siempre es el mismo. \citetitle[p.~3]{garcia2013simpromodel}
	\item Los tipos más comunes de simulación que existen son:
	      \begin{itemize}
		      \item \textbf{Simulación Continua}: Modela sistemas en los que los cambios ocurren continuamente a lo largo del tiempo (ej. modelos de crecimiento poblacional).
		      \item \textbf{Simulación de Eventos Discretos}: Modela sistemas en los que los cambios ocurren en puntos específicos en el tiempo (ej. sistemas de colas).
		      \item \small{\textbf{Simulación Basada en Agentes}: Modela sistemas donde múltiples entidades autónomas (agentes) interactúan entre sí (ej. mercados económicos, comportamiento social).} % TODO: agregar cita referenciando la diapositiva "TEMA 1" de la Ing. Gladys Machuca.
	      \end{itemize}
\end{enumerate}
\subsection*{Terminología Básica}
\begin{enumerate}
	\item Un modelo es una representación o abstracción de una situación real, el cual puede ser de varios tipos. Se diferencia de un sistema real en la medida que el objeto del modelo es poder comprender, predecir y controlar el comportamiento de un sistema dado.
	\item Definiciones:
	      \begin{itemize}
		      \item \textbf{Entidad:} Por lo general es \textit{la representación de los flujos de entrada y salida en un sistema;} al entrar a un sistema una entidad es el elemento responsable de que el estado del sistema cambie. \citetitle[p.~4]{garcia2013simpromodel}
		      \item \textbf{Atributo:} Es una \textit{característica de una entidad}. Son muy útiles
		            para diferenciar entidades sin necesidad de generar una nueva, y pueden adjudicarse al
		            momento de la creación de la entidad, asignarse o cambiarse durante el proceso. \citetitle[p.~6]{garcia2013simpromodel}
		      \item \textbf{Evento:} Es un \textit{cambio en el estado actual del sistema.} Existen dos tipos principales de eventos: \begin{itemize}
			            \item \textbf{Eventos actuales:} Aquellos que están sucediendo en el sistema en un momento dado.
			            \item \textbf{Eventos futuros:} Cambios que
			                  se presentaran en el sistema después del tiempo de simulación, de acuerdo con una pro­
			                  gramación específica.
		            \end{itemize} \citetitle[p.~4]{garcia2013simpromodel}
	      \end{itemize}
	\item \begin{itemize}
		      \item \textbf{Verificación:} Proceso en el cual, una vez identificado las distribuciones de probabilidad de las variables del modelo y se han implantado los supuestos acordados, se comprueba la propiedad de la programación del modelo, asegurándonos que todos los parámetros en la simulación funcionen correctamente. \citetitle[p.~13]{garcia2013simpromodel}
		      \item \textbf{Validación:} Proceso que consiste en pruebas simultáneas con información de entrada real para observar su comportamiento y analizar sus resultados. \citetitle[p.~14]{garcia2013simpromodel}
	      \end{itemize}
	\item Se define como el \textit{periodo que se modela}, que puede ser diferente del tiempo real. % Extraido de la diapositiva TEMA 1 de la Ing. Gladys.
\end{enumerate}
\subsection*{Conceptos Fundamentales de Simulación}
\begin{enumerate}[label*=\arabic*.]
	\item Los pasos principales en el proceso de desarrollo de un modelo de simulación son:
	      \begin{enumerate}[label*=\arabic*.]
		      \item \textbf{Definición del sistema bajo estudio:} En esta etapa es necesario conocer el sistema a modelar.
		            Para ello se requiere saber qué origina el estudio de simulación y establecer los supuestos del modelo: es conveniente definir con claridad
		            las variables de decisión del modelo, determinar las interacciones entre éstas,
		            y establecer con precisión los alcances y limitaciones que aquel podría llegar a tener.

		      \item \textbf{Generación del modelo de simulación base:} Una vez que se ha definido el
		            sistema en términos de un modelo conceptual, la siguiente etapa del estudio
		            consiste en la generación de un modelo de simulación base.
		            \\No es preciso que este modelo sea demasiado detallado, pues se requiere mucha más información
		            estadística sobre el comportamiento de las variables de decision del sistema.
		            La generación de este modelo es el primer reto para el programador de la simulación,
		            ya que debe traducir a un lenguaje de simulación la información que se
		            obtuvo en la etapa de definición del sistema, e incluir las interrelaciones de
		            todos los posibles subsistemas que existan en el problema a modelar.

		      \item \textbf{Recolección y análisis de datos:} En esta etapa se debe establecer qué información
		            es útil para la determinación de las distribuciones de probabilidad asociadas a
		            cada una de las variables aleatorias necesarias para la simulación.

		            Más
		            adelante se hará el análisis de los datos indispensables para asociar una distribución
		            de probabilidad a una variable aleatoria, así como las pruebas que se le
		            deben aplicar. Al finalizar la recolección y análisis de datos para todas las variables del modelo,
		            se tendrán las condiciones para \textit{\textbf{generar}} una versión \textit{\textbf{preliminar}}
		            del problema que se está simulando.

		      \item \textbf{Generación del modelo preliminar:} En esta etapa se integra la información
		            obtenida a partir del análisis de los datos, los supuestos del modelo y todos los
		            datos necesarios para crear un modelo lo más cercano posible a la realidad del
		            problema bajo estudio.

		            Al finalizar esta
		            etapa el \underline{modelo} está listo para su primera prueba: su \textit{\textbf{verificación}} o, en otras
		            palabras, la comparación con la realidad.

		      \item \textbf{Verificación del modelo:} Una vez que se han identificado las distribuciones de
		            probabilidad de las variables del modelo y se han implantado los supuestos
		            acordados, es necesario realizar un proceso de verificación de datos para comprobar la propiedad de
		            la programación del modelo, y comprobar que todos los
		            parámetros usados en la simulación funcionen correctamente.

		            Una vez que se ha completado la verificación, el modelo está listo para su
		            comparación con la realidad del problema que se esta modelando. A esta etapa
		            se le conoce también como \textit{\textbf{validación del modelo.}}

		      \item \textbf{Validación del modelo:} El proceso de validación del modelo consiste en realizar
		            una serie de pruebas simultáneas con información de entrada real para observar
		            su comportamiento y analizar sus resultados.
		            \\Si el problema bajo simulación involucra un proceso que se desea mejorar,
		            el modelo debe someterse a prueba con las condiciones actuales de operación,
		            lo que nos dará como resultado un comportamiento similar al que se presenta
		            realmente en nuestro proceso. \\Por otro lado, si se esta diseñando un nuevo proceso
		            la validación resulta mas complicada. Una manera de validar el modelo en
		            este caso, consiste en introducir algunos escenarios sugeridos por el cliente y
		            validar que el comportamiento sea congruente con las expectativas que se tienen
		            de acuerdo con la experiencia.

		      \item \textbf{Generación del modelo final:} Una vez que el modelo se ha validado, el analista
		            esta listo para realizar la simulación y estudiar el comportamiento del proceso.
		            En caso de que se desee comparar escenarios diferentes para un mismo problema,
		            este sera el modelo \textit{raíz}; en tal situación, el siguiente paso es la \textit{\textbf{definición de los escenarios a analizar.}}

		      \item \textbf{Determinación de los escenarios para el análisis:} Tras validar el modelo es
		            necesario acordar con el cliente los escenarios que se quieren analizar. Una
		            manera muy sencilla de determinarlos consiste en utilizar un escenario pesimista,
		            uno optimista y uno intermedio para la variable de respuesta mas importante.

		      \item \textbf{Análisis de sensibilidad:} Una vez que se obtienen los resultados de los escenarios
		            es importante realizar pruebas estadísticas que permitan comparar los escenarios
		            con los mejores resultados finales. Si dos de ellos tienen resultados
		            similares será necesario comparar sus intervalos de confianza respecto de la variable de respuesta final.
		            Si no hay intersección de intervalos podremos decir con
		            certeza estadística que los resultados no son iguales; sin embargo, si los intervalos
		            se sobreponen será imposible definir estadísticamente que una solución es mejor
		            que otra. Si se desea obtener un escenario "ganador”, será necesario realizar más
		            réplicas de cada modelo o incrementar el tiempo de simulación de cada corrida.
		            Con ello se busca acortar los intervalos de confianza de las soluciones finales y,
		            por consiguiente, incrementar la probabilidad de diferenciar las soluciones.

		      \item \textbf{Documentación del modelo, sugerencias y conclusiones:} Una vez realizado el
		            análisis de los resultados, es necesario efectuar toda la documentación del modelo.
		            Esta documentación es muy importante, pues permitirá el uso del modelo
		            generado en caso de que se requieran ajustes futuros. En ella se deben incluir los
		            supuestos del modelo, las distribuciones asociadas a sus variables, todos sus
		            alcances y limitaciones y, en general, la totalidad de las consideraciones de programación.
		            También es importante incluir sugerencias tanto respecto del uso del
		            modelo como sobre los resultados obtenidos, con el propósito de realizar un
		            reporte mas completo. Por último, deberán presentarse las conclusiones del
		            proyecto de simulación, a partir de las cuales es posible obtener los reportes
		            ejecutivos para la presentación final.

	      \end{enumerate} \citetitle[p.~12--15]{garcia2013simpromodel}
	      \newpage
	\item \textbf{Definición de estado:} Condición que guarda el sistema bajo estudio en un
	      momento de tiempo determinado; es como una fotografía de lo que está pasando en el
	      sistema en cierto instante. El estado del sistema se compone de variables o característi­
	      cas de operación puntuales (digamos el número de piezas que hay en el sistema en ese
	      momento), y de variables o características de operación acumuladas, o promedio (como
	      podría ser el tiempo promedio de permanencia de una entidad en el sistema, en una fila,
	      almacén o equipo). \citetitle[p.~4]{garcia2013simpromodel}

	\item \textbf{Corrida de simulación y la información que proporciona:}
	      La corrida de simulación es la acción de ejecutar la simulación,
	      habiendo establecido previamente una serie de parámetros.
	      Proporciona información estadística como el uso de localizaciones,
	      utilización de recursos, entre otros.
	      \begin{figure}[H]
		      \begin{center}
			      \includegraphics[width=0.95\textwidth]{figures/figure-6.17.png}
		      \end{center}
		      \caption{Figura 6.17: Ejemplo de información proporcionada por la corrida de simulación.}
	      \end{figure}\citetitle[p.~211--212]{garcia2013simpromodel}

	\item \textbf{La importancia de los números aleatorios en la simulación y cómo se generan:}
	      La importancia de los números aleatorios recae en que,
	      para poder realizar una simulación que incluya variabilidad dentro
	      de sus eventos, es preciso generar una serie de números
	      que sean aleatorios por sí mismos, y que su aleatoriedad
	      se extrapole al modelo de simulación que se está construyendo.
	      Se generan por medio de algoritmos determinísticos que requieren
	      parámetros de arranque. \citetitle[p.~22]{garcia2013simpromodel}

\end{enumerate}
\subsection*{Ventajas y Desventajas del Uso de la Simulación}
\begin{enumerate}
	\item \textbf{Principales ventajas del uso de la simulación en la toma de decisiones:} Algunas ventajas incluyen:
	      \begin{enumerate}
		      \item \underline{\textbf{Apoyo para prueba de conceptos:}} Mediante la simulación,
		            es posible realizar estudios piloto de diseños de proceso y productos
		            nuevos, con resultados rápidos y a un bajo costo. \citetitle[p.~2]{garcia2013simpromodel}
		      \item \underline{\textbf{Bajo riesgo:}} La simulación es
		            una muy buena herramienta para conocer el impacto en los procesos
		            sin necesidad de llevarlos a cabo en la realidad. \citetitle[p.~9]{garcia2013simpromodel}
		      \item \underline{\textbf{Visualización de procesos:}} En la actualidad
		            los paquetes de software para simulación tienden a ser mas sencillos,
		            lo que facilita su aplicación; gracias a las herramientas de animación
		            que forman parte de muchos de esos
		            paquetes es posible ver cómo se comportará un proceso una vez que sea mejorado. \citetitle[p.~10]{garcia2013simpromodel}
	      \end{enumerate}
	      Entre otros.
	\item \textbf{Limitaciones de la simulación como herramienta de análisis:}
	      \begin{enumerate}
		      \item \underline{\textbf{Optimización y costos:}} Aunque muchos paquetes
		            de software permiten obtener el mejor escenario a
		            partir de una combinación de variaciones posibles, la simulación \textbf{no
			            es una herramienta de optimizacion;} además, \textbf{puede ser costosa cuando
			            se quiere emplearla en problemas relativamente sencillos de resolver},
		            en lugar de utilizar soluciones analíticas que se
		            han desarrollado de manera especifica para ese tipo de casos. \citetitle[p.~10]{garcia2013simpromodel}
		      \item \underline{\textbf{Disponibilidad de tiempo:}} Se requiere bastante tiempo
		            ---por lo general meses--- para realizar un buen estudio de simulación;
		            por desgracia, no todos los analistas tienen la disposición
		            (o la oportunidad) de esperar ese tiempo para obtener una respuesta. \citetitle[p.~10]{garcia2013simpromodel}
		      \item \underline{\textbf{Conocimientos previos:}} Es preciso que el analista
		            domine el uso del paquete de simulación y que tenga
		            sólidos conocimientos de estadística para interpretar los resultados; adicionalmente,
		            ante una falta del mismo, los resultados de la simulación podrían llegar a ser
		            idealistas, mucho más de lo que sería posible en la vida real, debido a una mala
		            parametrización. \citetitle[p.~10]{garcia2013simpromodel}
	      \end{enumerate}
	\item \textbf{Situaciones donde es más beneficioso un modelo de simulación sobre un modelo analítico:}
	      \begin{enumerate}
		      \item \textbf{Sistemas Complejos:} Cuando el sistema es demasiado complejo para ser descrito adecuadamente mediante análisis analíticos.
		      \item \textbf{Evaluación Bajo Incertidumbre:} Se requiere evaluar el comportamiento del sistema bajo incertidumbre y variabilidad.
		      \item \textbf{Experimentación Segura:} Necesario experimentar con diferentes políticas o escenarios sin riesgos para el sistema real.
		      \item \textbf{Comprensión Visual:} Se busca una comprensión visual e interactiva del comportamiento del sistema.
	      \end{enumerate}% TODO: agregar cita referenciando la diapositiva "TEMA 1" de la Ing. Gladys Machuca.
	      \newpage
	\item \textbf{Identificación y mitigación de riesgos en un proyecto mediante la simulación:} La simulación permite identificar y mitigar riesgos en un proyecto al:
	      \begin{enumerate}
		      \item \textbf{Modelar Escenarios de Riesgo:} Crear modelos que representen diferentes escenarios de riesgo.
		      \item \textbf{Evaluar Impacto:} Analizar cómo los riesgos afectan el desempeño del proyecto.
		      \item \textbf{Desarrollar Estrategias de Mitigación:} Probar diferentes estrategias para mitigar riesgos y evaluar su efectividad.
		      \item \textbf{Monitoreo Continuo:} Utilizar la simulación para monitorear el progreso del proyecto y ajustar las estrategias de mitigación según sea necesario.
	      \end{enumerate}
\end{enumerate}

\printbibliography[title={Referencias}]
\end{document}
